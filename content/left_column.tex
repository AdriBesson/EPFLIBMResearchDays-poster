\vbox to \myposterheight{%
%----------------------------------------------------------------------------------------
%	OBJECTIVES
%----------------------------------------------------------------------------------------

\begin{block}{Introduction and objectives}
	
	\begin{itemize}
		\item The point spread function (PSF), \ie the response of a system to a point source, is a measure of the quality of an ultrasound~(US) imager
		\item PSF usuallycrudely estimated by assuming spatial-invariance, or estimated via simulation, coupled with reconstruction algorithms 
		\item Lack of an analytical formulation inhibits many applications ranging from optimization of apodization schemes, to array-design to and deconvolution algorithms
		\item We propose an analytical formulation of the PSF and use it for deconvolution of US images
	\end{itemize}
	
\end{block}
\vfill

%----------------------------------------------------------------------------------------
%	INTRODUCTION
%----------------------------------------------------------------------------------------

\begin{block}{Notations and data-model}
	
	\begin{columns} % Subdivide the first main column
		\begin{column}{.54\textwidth} % The first subdivided column within the first main column
			\begin{itemize}
				\item Notations:
				\begin{itemize}
					\item 1D probe composed of $N_{el}$ transducer elements located at $\bm{r}_i$
					\item $m_i\left(t\right)$ signal received at $i^{th}$ element
					\item Medium $\Omega \subset \mathbb{R}^2$ characterized by its tissue reflectivity function~(TRF) $\gamma \left(\vect{r}\right)$
					\item $b \left(\vect{r}, \vect{r}_i\right)$ spatially varying attenuation term  
					\item $h\left(t\right)$ is the elementary waveform
				\end{itemize}
			\end{itemize}
		\end{column}
		
		\begin{column}{.43\textwidth} % The second subdivided column within the first main column
			\centering
			\begin{figure}
				{\footnotesize
					\input{tikz/tikz_notations_scheme}}
				%\includegraphics[width=0.9\linewidth]{tikz_SPARS-crop.pdf}
				\caption{Standard setting for US imaging}
			\end{figure}
		\end{column}
	\end{columns} % End of the subdivision
	
	\begin{itemize}
		\item Data-model for a transducer located at $\vect{r}_i$:
		\begin{equation}
			m \left(\vect{r}_i, t\right) = \int \limits_{\Omega} b \left(\vect{r}', \vect{r}_i\right) h \left(t - \tau \left(\vect{r}', \vect{r}_i\right)\right) \gamma \left(\vect{r}'\right) d \vect{r}'
		\end{equation}
		where $\tau \left(\vect{r}', \vect{r}_i\right) = t_{TX} \left(\vect{r}'\right) + \frac{\| \vect{r}' - \vect{r}_i \|_2}{c}$ is the round trip time-of-flight of the US wave 
	\end{itemize}
	
\end{block}
\vfill 

%----------------------------------------------------------------------------------------
%	PSF
%----------------------------------------------------------------------------------------

\begin{block}{Model of the point-spread-function}
	\begin{itemize}
		\item Delay-and-sum~(DAS) estimate~(also called RF image) of the TRF:
		\begin{align}
		\label{eq_DAS_estimate}
			\ser{\gamma}{{DAS}} \left(\vect{r}\right) &= \sum \limits_{i=1}^{N_{el}} w_i m \left(\vect{r}_i, \tau \left(\vect{r}, \vect{r}_i\right)\right) \nonumber \\
			&= \int \limits_{\Omega} \left[\sum \limits_{i=1}^{N_{el}} w_i h \left(\tau \left(\vect{r}, \vect{r}_i\right) - \tau \left(\vect{r}', \vect{r}_i\right) \right)\right] \gamma\left(\vect{r}'\right) d \vect{r}'
		\end{align}
		where $b \left(\vect{r}', \vect{r}_i\right) = 1$
		\item One may deduce from Equation~\eqref{eq_DAS_estimate} 
		the kernel of the PSF:
		\begin{equation}
			\label{eq_PSF_estimate}
			PSF \left(\vect{r}, \vect{r}'\right) = \sum \limits_{i=1}^{N_{el}} w_i h \left(\tau \left(\vect{r}, \vect{r}_i\right) - \tau \left(\vect{r}', \vect{r}_i\right) \right), \; \forall \left(\vect{r}, \vect{r}'\right) \in \Omega^2
		\end{equation}
		\item Generic formulation:
		\begin{itemize}
			\item Grid-independent
			\item Compatible with any transmit scheme
			\item Compatible with any receive apodization 
		\end{itemize}
	\end{itemize}
\end{block}
\vfill

%----------------------------------------------------------------------------------------
%	VALIDATION OF THE PROPOSED MODEL
%----------------------------------------------------------------------------------------

\begin{block}{Validation of the proposed model}
	\begin{itemize}
		\item A cyst phantom is simulated
		\item The PSF model described in Equation~\eqref{eq_PSF_estimate} is generated with the following settings:
		\begin{itemize}
			\item The weights $w_i$ are set to 1
			\item The transmit scheme is plane wave with normal incidence: $t_{TX} \left(\vect{r}\right) = \frac{z}{c}$
		\end{itemize}
		\item The following experiments are run:
		\begin{enumerate}
			\item The PSF model is convolved with the phantoms in order to get a DAS estimate
			\item US raw-data are simulated on Field II~\cite{jensen1992} and reconstructed using standard DAS algorithm
		\end{enumerate}
	\end{itemize}
	\newlength{\CIRSFigWidth} \setlength{\CIRSFigWidth}{0.48\textwidth}
	\newlength{\CIRSFigHeight}
	\settoheight{\CIRSFigHeight}{\includegraphics[width=\CIRSFigWidth]{figures/Cyst_pht1.png}}
	\newlength{\pointFigWidth} \setlength{\CIRSFigWidth}{0.4\textwidth}
	\newlength{\pointFigHeight}
	\settoheight{\pointFigHeight}{\includegraphics[width=\CIRSFigWidth]{figures/Image_IUS_2017_1.jpg}}
	\begin{figure}
		% Maximum length
%		\subcaptionbox{Reference }{\includegraphics[height=\pointFigHeight]{figures/Image_IUS_2017_1.jpg}}
%		\hfill%
%		\subcaptionbox{CS reconstruction }{\includegraphics[height=\pointFigHeight]{figures/Image_IUS_2017_2.jpg}}
		\subcaptionbox{Field II simulation }{\includegraphics[height=\CIRSFigHeight]{figures/Cyst_pht1.png}}
		\hfill%
		\subcaptionbox{PSF model }{\includegraphics[height=\CIRSFigHeight]{figures/Cyst_pht2.png}}
		\hfill%
		\caption{B-mode image reconstructed from (a) Field II simulation and (b) the proposed PSF model}
	\end{figure}
\end{block}
}%